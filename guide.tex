%!TEX TS-program = xelatex
%!TEX encoding = UTF-8 Unicode

\documentclass[oneside,final]{article}
\usepackage[margin=2.5cm]{geometry}
\usepackage{ctex}
\usepackage{xltxtra}	% to use \XeLeTeX

% font setting
\usepackage[default,mdseries=Light,bfseries=Medium]{cjkfonts}

\newfontfamily\RobotoThin{Roboto}[
  Extension=.ttf,
  Path=/usr/local/texlive/2015/texmf-dist/fonts/truetype/google/roboto/,
  UprightFont=*-Thin]

\newfontfamily\RobotoLight{Roboto}[
  Extension=.ttf,
  Path=/usr/local/texlive/2015/texmf-dist/fonts/truetype/google/roboto/,
  UprightFont=*-Light]

\newfontfamily\RobotoRegular{Roboto}[
  Extension=.ttf,
  Path=/usr/local/texlive/2015/texmf-dist/fonts/truetype/google/roboto/,
  UprightFont=*-Regular]

\newfontfamily\RobotoMedium{Roboto}[
  Extension=.ttf,
  Path=/usr/local/texlive/2015/texmf-dist/fonts/truetype/google/roboto/,
  UprightFont=*-Medium]

\newfontfamily\RobotoBold{Roboto}[
  Extension=.ttf,
  Path=/usr/local/texlive/2015/texmf-dist/fonts/truetype/google/roboto/,
  UprightFont=*-Bold]

\newfontfamily\RobotoBlack{Roboto}[
  Extension=.ttf,
  Path=/usr/local/texlive/2015/texmf-dist/fonts/truetype/google/roboto/,
  UprightFont=*-Black]

\newfontfamily\SourceSansProExtraLight{SourceSansPro}[
  Extension=.otf,
  Path=/usr/local/texlive/2015/texmf-dist/fonts/opentype/adobe/sourcesanspro/,
  UprightFont=*-ExtraLight]

\newfontfamily\SourceSansProLight{SourceSansPro}[
  Extension=.otf,
  Path=/usr/local/texlive/2015/texmf-dist/fonts/opentype/adobe/sourcesanspro/,
  UprightFont=*-Light]

\newfontfamily\SourceSansProRegular{SourceSansPro}[
  Extension=.otf,
  Path=/usr/local/texlive/2015/texmf-dist/fonts/opentype/adobe/sourcesanspro/,
  UprightFont=*-Regular]

\newfontfamily\SourceSansProBold{SourceSansPro}[
  Extension=.otf,
  Path=/usr/local/texlive/2015/texmf-dist/fonts/opentype/adobe/sourcesanspro/,
  UprightFont=*-Bold]

\newfontfamily\SourceSansProSemibold{SourceSansPro}[
  Extension=.otf,
  Path=/usr/local/texlive/2015/texmf-dist/fonts/opentype/adobe/sourcesanspro/,
  UprightFont=*-Semibold]

\newfontfamily\SourceSansProBlack{SourceSansPro}[
  Extension=.otf,
  Path=/usr/local/texlive/2015/texmf-dist/fonts/opentype/adobe/sourcesanspro/,
  UprightFont=*-Black]


\setmainfont{Roboto}[
  Extension=.ttf,
  Path=/usr/local/texlive/2015/texmf-dist/fonts/truetype/google/roboto/,
  UprightFont=*-Light,
  BoldFont=*-Medium,
  ItalicFont=*-LightItalic,
  BoldItalicFont=*-MediumItalic]

\setsansfont{Roboto}[
  Extension=.ttf,
  Path=/usr/local/texlive/2015/texmf-dist/fonts/truetype/google/roboto/,
  UprightFont=*-Light,
  BoldFont=*-Medium,
  ItalicFont=*-LightItalic,
  BoldItalicFont=*-MediumItalic]

\setmonofont{SourceCodePro}[
  Extension=.otf,
  Path=/usr/local/texlive/2015/texmf-dist/fonts/opentype/adobe/sourcecodepro/,
  UprightFont=*-Light,
  BoldFont=*-Semibold,
  ItalicFont=*-LightIt,
  BoldItalicFont=*-SemiboldIt]

% hyperlink
\usepackage[colorlinks=true,linkcolor=blue]{hyperref}

% for code example
\usepackage{listings}
\usepackage{color}

\definecolor{mygreen}{rgb}{0,0.6,0}
\definecolor{mygray}{rgb}{0.5,0.5,0.5}
\definecolor{mymauve}{rgb}{0.58,0,0.82}

\lstset{ %
  backgroundcolor=\color{white},   % choose the background color; you must add \usepackage{color} or \usepackage{xcolor}
  basicstyle=\footnotesize\ttfamily,        % the size of the fonts that are used for the code
  breakatwhitespace=false,         % sets if automatic breaks should only happen at whitespace
  breaklines=true,                 % sets automatic line breaking
  captionpos=b,                    % sets the caption-position to bottom
  commentstyle=\color{mygreen},    % comment style
%  deletekeywords={...},            % if you want to delete keywords from the given language
  escapeinside={\%*}{*)},          % if you want to add LaTeX within your code
  extendedchars=true,              % lets you use non-ASCII characters; for 8-bits encodings only, does not work with UTF-8
  frame=single,                    % adds a frame around the code
  keepspaces=true,                 % keeps spaces in text, useful for keeping indentation of code (possibly needs columns=flexible)
%  keywordstyle=\color{blue},       % keyword style
%  language=Octave,                 % the language of the code
%  otherkeywords={*,...},           % if you want to add more keywords to the set
  numbers=left,                    % where to put the line-numbers; possible values are (none, left, right)
  numbersep=5pt,                   % how far the line-numbers are from the code
  numberstyle=\tiny\color{mygray}, % the style that is used for the line-numbers
  rulecolor=\color{black},         % if not set, the frame-color may be changed on line-breaks within not-black text (e.g. comments (green here))
  showspaces=false,                % show spaces everywhere adding particular underscores; it overrides 'showstringspaces'
  showstringspaces=false,          % underline spaces within strings only
  showtabs=false,                  % show tabs within strings adding particular underscores
  stepnumber=1,                    % the step between two line-numbers. If it's 1, each line will be numbered
  stringstyle=\color{mymauve},     % string literal style
  tabsize=2                        % sets default tabsize to 2 spaces
%  title=\lstname                   % show the filename of files included with \lstinputlisting; also try caption instead of title
}

\linespread{1.5}

\begin{document}

\title{cjkfonts 宏包}
\author{Freeman Zhang}
\date{\today{} v0.1}

\maketitle

\section{简介}

\href{https://github.com/zhanggyb/cjkfonts}{cjkfonts} 是一个简单的\XeLaTeX{}宏包,可用于排版包含中日韩(CJK)文字的文档。其将\href{https://github.com/adobe-fonts/source-han-sans}{思源黑体}设置为默认CJK字体,排版效果如下:

\begin{center}
  Normal Text {\SourceHanSansSC 正常文本} \\
  \vspace{1em}
  \textbf{Bold Text {\SourceHanSansSC 粗体文本}} \\
  \vspace{1em}
  \textit{Italic Text {\SourceHanSansSC 斜体文本}} \\
  \vspace{1em}
  \textbf{\textit{BoldItalic Text {\SourceHanSansSC 粗斜体文本}}}
\end{center}

\section{使用}

\subsection{签出源码}

\begin{lstlisting}[language=sh]
$ git clone https://github.com/zhanggyb/cjkfonts
\end{lstlisting}

\subsection{准备字体文件}

运行bootstrap.sh脚本来自动将需要的字体文件下载到fonts目录,这一脚本依赖git和curl。

\begin{lstlisting}[language=sh]
$ ./bootstrap.sh
\end{lstlisting}

也可以自行下载并放在 fonts 目录。

\subsection{复制文件}

直接将cjkfonts.sty以及fonts目录复制到你的文档目录。

\subsection{使用宏包}

与其它\LaTeX{}宏包一样,引入cjkfonts宏包只需要在导言区使用:

\begin{lstlisting}[language=TeX]
\usepackage{cjkfonts}
\end{lstlisting}

使用path选项可以选择其它字体所在目录:

\begin{lstlisting}[language=TeX]
\usepackage[path=<mypath>]{cjkfonts}
\end{lstlisting}

使用default选项在加载宏包时设置CJK字体设置为\href{https://github.com/adobe-fonts/source-han-sans}{思源黑体}:

\begin{lstlisting}[language=TeX]
\usepackage[default]{cjkfonts}
\end{lstlisting}


为了更好的屏幕阅读体验,建议将mdseries和bfseries设置为较细的字体:

\begin{lstlisting}
\usepackage[mdseries=Light,bfseries=Medium]{cjkfonts}
\end{lstlisting}

\subsection{示例}

这里是一个简单的示例:

\begin{lstlisting}[language=TeX]
%!TEX TS-program = xelatex
%!TEX encoding = UTF-8 Unicode

\documentclass{article}
\usepackage[default,mdseries=Light,bfseries=Medium]{cjkfonts}

\begin{document}
\begin{center}
  Normal Text 正常文本 \\
  \vspace{1em}
  \textbf{Bold Text 粗体文本} \\
  \vspace{1em}
  \textit{Italic Text 斜体文本} \\
  \vspace{1em}
  \textbf{\textit{BoldItalic Text 粗斜体文本}}
\end{center}
\end{document}
\end{lstlisting}

\textbf{注意:}需要使用\XeLaTeX{}排版引擎来编译源码。

\section{手册}

\subsection{选项}

\begin{center}
\begin{tabular}{ c l }
	\textbf{path=<font dir>} & 设置\href{https://github.com/adobe-fonts/source-han-sans}{思源黑体}字体文件路径为<font dir>指定的路径 \\
	\textbf{default} & 设置默认CJK字体为\href{https://github.com/adobe-fonts/source-han-sans}{思源黑体} \\
	\textbf{mdseries=<weight>} & 设置常规字体字重,<weight>只能从以下表格选择一个 \\
	\textbf{bfseries=<weight>} & 设置粗字体字重,<weight>只能从以下表格选择一个 \\
\end{tabular}	
\end{center}

\begin{center}
\begin{tabular}{ | c | c | }
\hline
weight 可取值 & 样例 \\
\hline
ExtraLight & {\SourceHanSansSCExtraLight 朝辞白帝彩云间,千里江陵一日还。} \\
Light      & {\SourceHanSansSCLight 朝辞白帝彩云间,千里江陵一日还。} \\
Normal     & {\SourceHanSansSCNormal 朝辞白帝彩云间,千里江陵一日还。} \\
Regular    & {\SourceHanSansSCRegular 朝辞白帝彩云间,千里江陵一日还。} \\
Medium     & {\SourceHanSansSCMedium 朝辞白帝彩云间,千里江陵一日还。} \\
Bold       & {\SourceHanSansSCBold 朝辞白帝彩云间,千里江陵一日还。} \\
Heavy      & {\SourceHanSansSCHeavy 朝辞白帝彩云间,千里江陵一日还。} \\
\hline
\end{tabular}	
\end{center}

\subsection{字体族}

使用同名的字体族来选择字体:

\begin{itemize}
\item {\verb!\SourceHanSansSC!}
\item {\verb!\SourceHanSansSCExtraLight!}
\item {\verb!\SourceHanSansSCLight!}
\item {\verb!\SourceHanSansSCNormal!}
\item {\verb!\SourceHanSansSCRegular!}
\item {\verb!\SourceHanSansSCMedium!}
\item {\verb!\SourceHanSansSCBold!}
\item {\verb!\SourceHanSansSCHeavy!}
\end{itemize}

\subsection{CJK支持}

\href{https://github.com/adobe-fonts/source-han-sans}{思源黑体}已经支持CJK文本。

例如:

\begin{center}
  \begin{tabular}{ r l }
    \textbf{简体中文} & 每个人生来平等,享有相同的地位和权利。 \\
    \textbf{繁體中文} & 每個人生來平等,享有相同的地位和權利。 \\
    \textbf{한국의} & 모두가 동일한 태어나 같은 지위와 권리 를 가지고 있다。\\
	\textbf{日本語} & すべての人間は自由であり、かつ、尊厳と権利とについて平等である。\\
  \end{tabular}
\end{center}

\section{已知问题}

\begin{enumerate}
	\item bootstrap.sh 脚本只在OS X和Linux系统上测试过。
\end{enumerate}

\section{中文排版建议}

将\href{https://github.com/adobe-fonts/source-han-sans}{思源黑体}与其它英文字体配合使用可以得到很好的排版效果,推荐Adobe开源字体或Google Roboto字体。
这些字体已经随着TeX Live 2015版本发布,安装后在TEXDIR目录可以找到字体文件。

\subsection{与Adobe字体族配合使用}

\href{https://github.com/adobe-fonts/source-han-sans}{思源黑体}与Adobe Source Sans Pro英文字体配合:

\begin{center}
	{\SourceSansProExtraLight Source Sans Pro Extra Light}{\SourceHanSansSCExtraLight 与思源黑体}{\SourceSansProExtraLight Extra Light}{\SourceHanSansSCExtraLight 混排效果示例}\\
	{\SourceSansProLight Source Sans Pro Light}{\SourceHanSansSCLight 与思源黑体}{\SourceSansProLight Light}{\SourceHanSansSCLight 混排效果示例}\\
	{\SourceSansProRegular Source Sans Pro Regular}{\SourceHanSansSCRegular 与思源黑体}{\SourceSansProRegular Regular}{\SourceHanSansSCRegular 混排效果示例}\\
	{\SourceSansProSemibold Source Sans Pro Semibold}{\SourceHanSansSCMedium 与思源黑体}{\SourceSansProSemibold Medium}{\SourceHanSansSCMedium 混排效果示例}\\
	{\SourceSansProBold Source Sans Pro Bold}{\SourceHanSansSCBold 与思源黑体}{\SourceSansProBold Bold}{\SourceHanSansSCBold 混排效果示例}\\
	{\SourceSansProBlack Source Sans Pro Black}{\SourceHanSansSCHeavy 与思源黑体}{\SourceSansProBlack Heavy}{\SourceHanSansSCHeavy 混排效果示例}
\end{center}

%\begin{lstlisting}[language=TeX]
%\setmainfont{SourceSansPro}[
%  Extension=.otf,
%  Path=/usr/local/texlive/2015/texmf-dist/fonts/opentype/adobe/sourcesanspro/,
%  UprightFont=*-Light,
%  BoldFont=*-Semibold,
%  ItalicFont=*-LightIt,
%  BoldItalicFont=*-SemiboldIt]
%
%\setsansfont{SourceSerifPro}[
%  Extension=.otf,
%  Path=/usr/local/texlive/2015/texmf-dist/fonts/opentype/adobe/sourceserifpro/,
%  UprightFont=*-Light,
%  BoldFont=*-Semibold,
%  ItalicFont=*-Light,
%  BoldItalicFont=*-Semibold,
%  ItalicFeatures=FakeSlant,
%  BoldItalicFeatures=FakeSlant]
%
%\setmonofont{SourceCodePro}[
%  Extension=.otf,
%  Path=/usr/local/texlive/2015/texmf-dist/fonts/opentype/adobe/sourcecodepro/,
%  UprightFont=*-Light,
%  BoldFont=*-Semibold,
%  ItalicFont=*-LightIt,
%  BoldItalicFont=*-SemiboldIt]
%\end{lstlisting}

\subsection{与Google Roboto英文字体配合}

\href{https://github.com/adobe-fonts/source-han-sans}{思源黑体}与Google Roboto英文字体配合:

\begin{center}
	{\RobotoThin Roboto Thin}{\SourceHanSansSCExtraLight 与思源黑体}{\RobotoThin Extra Light}{\SourceHanSansSCExtraLight 混排效果示例}\\
	{\RobotoLight Roboto Light}{\SourceHanSansSCLight 与思源黑体}{\RobotoLight Light}{\SourceHanSansSCLight 混排效果示例}\\
	{\RobotoRegular Roboto Regular}{\SourceHanSansSCRegular 与思源黑体}{\RobotoRegular Regular}{\SourceHanSansSCRegular 混排效果示例}\\
	{\RobotoMedium Roboto Medium}{\SourceHanSansSCMedium 与思源黑体}{\RobotoMedium Medium}{\SourceHanSansSCMedium 混排效果示例}\\
	{\RobotoBold Roboto Bold}{\SourceHanSansSCBold 与思源黑体}{\RobotoBold Bold}{\SourceHanSansSCBold 混排效果示例}\\
	{\RobotoBlack Roboto Black}{\SourceHanSansSCHeavy 与思源黑体}{\RobotoBlack Black}{\SourceHanSansSCHeavy 混排效果示例}
\end{center}

%\begin{lstlisting}[language=TeX]
%\setmainfont{Roboto}[
%  Extension=.ttf,
%  Path=/usr/local/texlive/2015/texmf-dist/fonts/truetype/google/roboto/,
%  UprightFont=*-Light,
%  BoldFont=*-Medium,
%  ItalicFont=*-LightItalic,
%  BoldItalicFont=*-MediumItalic]
%
%\setsansfont{Roboto}[
%  Extension=.ttf,
%  Path=/usr/local/texlive/2015/texmf-dist/fonts/truetype/google/roboto/,
%  UprightFont=*-Light,
%  BoldFont=*-Medium,
%  ItalicFont=*-LightItalic,
%  BoldItalicFont=*-MediumItalic]
%
%\setmonofont{SourceCodePro}[
%  Extension=.otf,
%  Path=/usr/local/texlive/2015/texmf-dist/fonts/opentype/adobe/sourcecodepro/,
%  UprightFont=*-Light,
%  BoldFont=*-Semibold,
%  ItalicFont=*-LightIt,
%  BoldItalicFont=*-SemiboldIt]
%\end{lstlisting}

\section{代码实现}

cjkfonts的程序代码非常简单,只是利用\href{https://github.com/ctex-org/ctex-kit}{xeCJK}宏包设置排版字体:

\lstinputlisting[language=TeX]{cjkfonts.sty}

\end{document}
