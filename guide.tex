%!TEX TS-program = xelatex
%!TEX encoding = UTF-8 Unicode

\documentclass[oneside,final]{article}
\usepackage[margin=2.5cm]{geometry}
\usepackage{ctex}
\usepackage{cjkfonts}
\usepackage{xltxtra}	% to use \XeLeTeX

% hyperlink
\usepackage[colorlinks=true,linkcolor=blue]{hyperref}

% for code example
\usepackage{listings}
\usepackage{color}

\definecolor{mygreen}{rgb}{0,0.6,0}
\definecolor{mygray}{rgb}{0.5,0.5,0.5}
\definecolor{mymauve}{rgb}{0.58,0,0.82}

\lstset{ %
  backgroundcolor=\color{white},   % choose the background color; you must add \usepackage{color} or \usepackage{xcolor}
  basicstyle=\footnotesize,        % the size of the fonts that are used for the code
  breakatwhitespace=false,         % sets if automatic breaks should only happen at whitespace
  breaklines=true,                 % sets automatic line breaking
  captionpos=b,                    % sets the caption-position to bottom
  commentstyle=\color{mygreen},    % comment style
%  deletekeywords={...},            % if you want to delete keywords from the given language
  escapeinside={\%*}{*)},          % if you want to add LaTeX within your code
  extendedchars=true,              % lets you use non-ASCII characters; for 8-bits encodings only, does not work with UTF-8
  frame=single,                    % adds a frame around the code
  keepspaces=true,                 % keeps spaces in text, useful for keeping indentation of code (possibly needs columns=flexible)
%  keywordstyle=\color{blue},       % keyword style
%  language=Octave,                 % the language of the code
%  otherkeywords={*,...},           % if you want to add more keywords to the set
  numbers=left,                    % where to put the line-numbers; possible values are (none, left, right)
  numbersep=5pt,                   % how far the line-numbers are from the code
  numberstyle=\tiny\color{mygray}, % the style that is used for the line-numbers
  rulecolor=\color{black},         % if not set, the frame-color may be changed on line-breaks within not-black text (e.g. comments (green here))
  showspaces=false,                % show spaces everywhere adding particular underscores; it overrides 'showstringspaces'
  showstringspaces=false,          % underline spaces within strings only
  showtabs=false,                  % show tabs within strings adding particular underscores
  stepnumber=1,                    % the step between two line-numbers. If it's 1, each line will be numbered
  stringstyle=\color{mymauve},     % string literal style
  tabsize=2                        % sets default tabsize to 2 spaces
%  title=\lstname                   % show the filename of files included with \lstinputlisting; also try caption instead of title
}

\linespread{1.5}

\begin{document}

\title{cjkfonts 宏包}
\author{Freeman Zhang}
\date{\today{} v0.1}

\maketitle

\section{简介}

\href{https://github.com/zhanggyb/cjkfonts}{cjkfonts} 是一个简单的\XeLaTeX{}宏包,其中预设了一些字体配置,可用于排版包含中日韩(CJK)文字的文档。

这一宏包将\href{https://github.com/adobe-fonts/source-han-sans}{思源黑体}设置为默认西文和CJK字体,包括正文、无衬线、等宽,并且默认使用细字体来获得更好的屏幕阅读体验。排版效果如下:

\begin{center}
  Normal Text 正常文本 \\
  \vspace{1em}
  \textbf{Bold Text 粗体文本} \\
  \vspace{1em}
  \textit{Italic Text 斜体文本} \\
  \vspace{1em}
  \textbf{\textit{BoldItalic Text 粗斜体文本}}
\end{center}

除了\href{https://github.com/adobe-fonts/source-han-sans}{思源黑体},这一宏包还引用了其它一些出色的开源字体方案来丰富排版效果,其使用的所有字体有:

\begin{itemize}
\item {\SourceHanSansSC \href{https://github.com/adobe-fonts/source-han-sans}{Adobe思源黑体}}\footnote{简体中文字体}
\item {\Roboto \href{https://github.com/google/roboto}{Google Roboto}}  
\item {\RobotoCondensed \href{https://github.com/google/roboto}{Google RobotoCondensed}}
\item {\SourceSansPro \href{https://github.com/adobe-fonts/source-sans-pro}{Adobe Source Sans Pro}}  
\item {\SourceSerifPro \href{https://github.com/adobe-fonts/source-serif-pro}{Adobe Serif Pro}}
\item {\SourceCodePro \href{https://github.com/adobe-fonts/source-code-pro}{Adobe Source Code Pro}}
\end{itemize}

\section{使用}

\subsection{签出源码}

\begin{lstlisting}[language=sh]
$ git clone https://github.com/zhanggyb/cjkfonts
\end{lstlisting}

\subsection{准备字体文件}

运行{\SourceCodePro bootstrap.sh}脚本来自动将需要的字体文件下载到
{\SourceCodePro fonts}目录,这一脚本依赖git和curl。

\begin{lstlisting}[language=sh]
$ ./bootstrap.sh
\end{lstlisting}

也可以自行下载并放在 fonts 目录。

\subsection{复制文件}

直接将{\SourceCodePro cjkfonts.sty}以及{\SourceCodePro fonts}目录复制到你的文档目录。

\subsection{使用宏包}

与其它\LaTeX{}宏包一样,引入{\SourceCodePro cjkfonts}宏包只需要在导言区使用:

\begin{lstlisting}
\usepackage{cjkfonts}
\end{lstlisting}

为了更好的屏幕阅读体验,cjkfonts 默认使用纤细\footnote{标注为Thin或者Light的字体文件,例如SourceSansPro-Light.otf}的字体,如果你需要用正常粗细的效果,可以使用选项:

\begin{lstlisting}
\usepackage[medium]{cjkfonts}
\end{lstlisting}

\subsection{示例}

这里是一个最简单的示例:

\begin{lstlisting}[language=TeX]
%!TEX TS-program = xelatex
%!TEX encoding = UTF-8 Unicode

\documentclass{article}
\usepackage{cjkfonts}

\begin{document}
\begin{center}
  Normal Text 正常文本 \\
  \vspace{1em}
  \textbf{Bold Text 粗体文本} \\
  \vspace{1em}
  \textit{Italic Text 斜体文本} \\
  \vspace{1em}
  \textbf{\textit{BoldItalic Text 粗斜体文本}}
\end{center}
\end{document}
\end{lstlisting}

\textbf{注意:}需要使用\XeLaTeX{}排版引擎来编译源码。

\section{手册}

\subsection{命令}

\subsubsection{设置字体目录}

使用 \verb!\setfontdir{<path>}!来设置字体文件所在目录,默认为同级的 \verb!fonts/!。

\subsection{默认字体}

\href{https://github.com/adobe-fonts/source-han-sans}{思源黑体}

\subsection{字体族}

使用同名的字体族来选择字体:

\begin{itemize}
	\item {\SourceCodePro \verb!\Roboto!}
	\item {\SourceCodePro \verb!\RobotoCondensed!}
	\item {\SourceCodePro \verb!\SourceSansPro!}
	\item {\SourceCodePro \verb!\SourceSerifPro!}
	\item {\SourceCodePro \verb!\SourceCodePro!}
	\item {\SourceCodePro \verb!\SourceHanSansSC!}
\end{itemize}

例如:

{\SourceCodePro \verb!{\Roboto Now use Roboto font}!} --- {\Roboto Now use Roboto font}

\subsection{CJK支持}

\href{https://github.com/adobe-fonts/source-han-sans}{思源黑体}已经支持CJK文本。

例如:

\begin{center}
  \begin{tabular}{ r l }
    \textbf{简体中文} & 每个人生来平等,享有相同的地位和权利。 \\
    \textbf{繁體中文} & 每個人生來平等,享有相同的地位和權利。 \\
    \textbf{한국의} & 모두가 동일한 태어나 같은 지위와 권리 를 가지고 있다。\\
	\textbf{日本語} & すべての人間は自由であり、かつ、尊厳と権利とについて平等である。\\
  \end{tabular}
\end{center}

\subsection{程序代码}

程序代码使用 Adobe Code Pro 字体:

\begin{lstlisting}[language=C]
#include <stdio.h>
#define N 10

int main()
{
    int i;

    // Line comment.
    puts("Hello world!");

    for (i = 0; i < N; i++)
    {
        puts("LaTeX is also great for programmers!");
    }

    return 0;
}
\end{lstlisting}

\section{已知问题}

\begin{enumerate}
	\item bootstrap.sh 脚本只在OS X和Linux系统上测试过。
\end{enumerate}

\section{代码实现}

cjkfonts的程序代码非常简单,只是利用fontspec和xeCJK设置排版字体:

\lstinputlisting[language=TeX]{cjkfonts.sty}

\end{document}
